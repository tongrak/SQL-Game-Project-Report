\documentclass[semifinal]{cpecmu}

%% This is a sample document demonstrating how to use the CPECMU
%% project template. If you are having trouble, see "cpecmu.pdf" for
%% documentation.

\projectNo{S012-2/65}
\acadyear{2020}

\titleTH{เกมเพื่อการศึกษาภาษา SQL}
\titleEN{Game for learning SQL}

\author{นายต้องรัก เรือนโต}{Tongrak Ruento}{630610730}
\author{นายแทนหทัย คลังมาเจริญ}{Tanhatai Klungmajaroen}{630610732}

\cpeadvisor{chinawat}
\cpecommittee{karn}
\committee{รศ.ดร.\,จักรพงศ์ นาทวิชัย}{Assoc.\,Prof.\,Juggapong Natwichai, Ph.D.}

%% Some possible packages to include:
\usepackage[final]{graphicx} % for including graphics

%% Add bookmarks and hyperlinks in the document.
\PassOptionsToPackage{hyphens}{url}
\usepackage[colorlinks=true,allcolors=Blue4,citecolor=red,linktoc=all]{hyperref}
\def\UrlLeft#1\UrlRight{$#1$}

%% Set up commenting
\iffinal
  \usepackage[disabled]{authcomments}
\else
  \usepackage{authcomments}
\fi
\newcommenter{CI}{0.0,0.5625,0.0}  % green

%% Needed just by this example, but maybe not by most reports
\usepackage{afterpage} % for outputting
\usepackage{pdflscape} % for landscape figures and tables. 

%% Some other useful packages. Look these up to find out how to use
%% them.
% \usepackage{natbib}    % for author-year citation styles
% \usepackage{txfonts}
% \usepackage{appendix}  % for appendices on a per-chapter basis
% \usepackage{xtab}      % for tables that go over multiple pages
% \usepackage{subfigure} % for subfigures within a figure
% \usepackage{pstricks,pdftricks} % for access to special PostScript and PDF commands
% \usepackage{nomencl}   % if you have a list of abbreviations

%% if you're having problems with overfull boxes, you may need to increase
%% the tolerance to 9999
% \tolerance=9999

\bibliographystyle{plain}
% \bibliographystyle{IEEEbib}

% \renewcommand{\topfraction}{0.85}
% \renewcommand{\textfraction}{0.1}
% \renewcommand{\floatpagefraction}{0.75}

%% Example for glossary entry
%% Need to use glossary option
%% See glossaries package for complete documentation.
\ifglossary
  \newglossaryentry{lorem ipsum}{
    name=lorem ipsum,
    description={derived from Latin dolorem ipsum, translated as ``pain itself''}
  }
\fi

%% Uncomment this command to preview only specified LaTeX file(s)
%% imported with \include command below.
%% Any other file imported via \include but not specified here will not
%% be previewed.
%% Useful if your report is large, as you might not want to build
%% the entire file when editing a certain part of your report.
% \includeonly{chapters/intro,chapters/background}

\begin{document}
\maketitle
\makesignature

\ifproject
\begin{abstractTH}
เขียนบทคัดย่อของโครงงานที่นี่

ณ ปัจจุบันมีเกมจำนวนน้อยที่มีเป้าหมายในการสอนภาษาโปรแกรมที่มีความจำเป็นในกลุ่มต่างๆนอกเหนือกลุ่มนักพัฒนาซอฟต์แวร์เช่นภาษา SQL ซึ่งถูกใช้งานอย่างแพร่หลายในการจัดการฐานข้อมูลต่างๆในชีวิตประจำวัน ถึงกระนั้นเกมที่ออกแบบมาเพื่อสอนภาษา SQL เช่น SQL Murder Mystery นั้นมีการตอบโต้และการแสดงออกต่างๆอยู่ในรูปแบบของตำราที่สามารถโต้ตอบได้ ซึ่งรูปแบบของเกมดังกล่าวถูกมองว่าไม่เป็นมิตรและไม่น่าดึงดูดสำหรับผู้ที่สนใจจะศึกษา(หรือผู้ใช้ทั่วไป)และทำให้ผู้ใช้งานเล่นเกมไม่ถึงจุดสุดท้าย กล่าวคือผู้ใช้งานจะไม่ได้รับความรู้ทั้งหมดที่ควรได้เนื่องจากความเบื่อหน่าย ทางกลุ่มจึงมีความสนใจในการพัฒนาเกมสำหรับการสอนภาษา SQL ในรูปแบบที่ผู้เล่นเกม ทั่วและผู้สนใจจะศึกษาสามารถเข้าถึงได้ กล่าวคือรูปแบบของเกมนั้นจะต้องเป็นมิตรกับผู้ใช้ทั่วไปแต่ยังคงความน่าสนใจในกลุ่มผู้ใช้ที่เล่นเกมเป็นประจำ และน่าดึงดูดเพื่อให้ผู้ใช้งานสามารถได้ศึกษาจนจบได้
\end{abstractTH}

\begin{abstract}
The abstract would be placed here. It usually does not exceed 350 words
long (not counting the heading), and must not take up more than one (1) page
(even if fewer than 350 words long).

Make sure your abstract sits inside the \texttt{abstract} environment.
\end{abstract}

\iffalse
\begin{dedication}
This document is dedicated to all Chiang Mai University students.

Dedication page is optional.
\end{dedication}
\fi % \iffalse

\begin{acknowledgments}
Your acknowledgments go here. Make sure it sits inside the
\texttt{acknowledgment} environment.

\acksign{2020}{5}{25}
\end{acknowledgments}%
\fi % \ifproject

\contentspage

\ifproject
\figurelistpage

\tablelistpage
\fi % \ifproject

% \abbrlist % this page is optional

% \symlist % this page is optional

% \preface % this section is optional


\pagestyle{empty}\cleardoublepage
\normalspacing \setcounter{page}{1} \pagenumbering{arabic} \pagestyle{cpecmu}

\include{chapters/intro}
\include{chapters/background}
\include{chapters/approach}
\include{chapters/eval}
\ifproject
\include{chapters/conclusion}
\fi

\bibliography{sampleReport}

\ifproject
\normalspacing
\appendix
\include{chapters/appendix}

%% Display glossary (optional) -- need glossary option.
\ifglossary\glossarypage\fi

%% Display index (optional) -- need idx option.
\ifindex\indexpage\fi

\begin{biosketch}
\begin{center}
  \includegraphics[width=1.5in]{mugshot.jpg}
\end{center}
Your biosketch goes here. Make sure it sits inside
the \texttt{biosketch} environment.
\end{biosketch}
\fi % \ifproject
\end{document}
